% Created 2017-10-09 Mon 16:03
% Intended LaTeX compiler: pdflatex
\documentclass[11pt]{article}
\usepackage[utf8]{inputenc}
\usepackage[T1]{fontenc}
\usepackage{graphicx}
\usepackage{grffile}
\usepackage{longtable}
\usepackage{wrapfig}
\usepackage{rotating}
\usepackage[normalem]{ulem}
\usepackage{amsmath}
\usepackage{textcomp}
\usepackage{amssymb}
\usepackage{capt-of}
\usepackage{hyperref}
\author{Martin Marier}
\date{}
\title{Instructions pour utiliser l'éponge.}
\hypersetup{
 pdfauthor={Martin Marier},
 pdftitle={Instructions pour utiliser l'éponge.},
 pdfkeywords={},
 pdfsubject={},
 pdfcreator={Emacs 25.3.1 (Org mode 9.1)}, 
 pdflang={Frenchb}}
\begin{document}

\maketitle

\section{Installer SuperCollider et les plugins}
\label{sec:orge84f0c0}
La version la plus récente de SC est disponible ici:
\url{http://supercollider.github.io/download} Nous utilisons actuellement la
version 3.8.0.

Les plugins sont disponibles ici:
\url{https://github.com/supercollider/sc3-plugins/releases} Nous utilisons
actuellement la version 3.8.0.

Une fois l'archive ZIP dépaquetée, le dossier \texttt{SC3plugins} et tout
son contenu doit être placé dans ce dossier (sur Mac):

\texttt{/Users/yourUserName/Library/Application Support/SuperCollider/Extensions}

Créez le dossier \texttt{Extensions} au besoin.

Pour connaître le dossier dans lequel vous devez déposer les
\texttt{SC3plugins} sur une autre plateforme (Windows ou Linux), démarrez
SuperCollider et cliquez sur l'item \emph{Open user support directory} du
menu \emph{File}.

\section{Installer tous les Quarks nécessaires}
\label{sec:orgd7692ea}
La liste des quarks nécessaires est dans le fichier
\texttt{scCode/installation.scd}

Vous devez ouvrir ce document avec SuperCollider et exécuter le bloc de code
qui est entre parenthèses.

Pour exécuter le bloc, placez le curseur sur la première parenthèse et
appuyez sur Command+Enter (ou Ctrl+Enter).

\section{Installer les mmExtensions}
\label{sec:org8657fd7}
Téléchargez les mmExtensions ici: \url{https://github.com/marierm/mmExtensions}

Le fichier ZIP doit être dépaqueté dans ce dossier (sur mac):

\texttt{/Users/yourUserName/Library/Application Support/SuperCollider/Extensions}

Créez le dossier \texttt{Extensions} au besoin.

\section{Configurer l'éponge}
\label{sec:orgb2519c4}
Lors de sa mise sous tension, l'éponge tente de se connecter au
dernier point d'accès WiFi (un routeur sans fil) auquel elle s'est
connectée.

Si ce point d'accès n'est pas disponible, l'éponge devient elle-même
un point d'accès dont le nom est \texttt{wifi101-XXXX}, où \texttt{XXXX}
représente les quatre derniers caractères de l'adresse MAC de
l'éponge.  L'adresse MAC -- unique pour chaque appareil pouvant se
connecter à un résea WiFi -- est imprimée sur le module WiFi de
l'éponge.

Il faudra alors se connecter à ce réseau Wifi et, de là, accéder à
la page web servie par l'éponge à l'adresse \url{http://192.168.1.1} .
Sur cette page, entrez le nom de votre point d'accès et le mot de
passe correspondant.  L'éponge se connectera alors automatiquement
au point d'accès que vous avez entré.  (À partir de ce moment, la
page \url{http://192.168.1.1} n'est plus accessible.  Il est donc normal
que le navigateur utilisé affiche un message d'erreur.)

\section{Démarrer l'éponge en mode solo}
\label{sec:org889919d}
Ouvrez le fichier scCode/solo.scd dans SuperCollider et exécuter le bloc de code
qui est entre parenthèses.

Pour exécuter le bloc, placez le curseur sur la première parenthèse
et appuyez sur Command+Enter (ou Ctrl+Enter).

L'éponge est prête quand le message \texttt{sponge\_solo ready.} apparaît dans la fenêtre
\texttt{Post} de SuperCollider.

\section{Messages OSC}
\label{sec:orgc949d6d}
(Vous n'avez pas besoin de comprendre tout ça.)

L'éponge envoie des messages OSC en \emph{multicast} à l'adresse
\texttt{224.0.0.1} sur le port 50501, à partir du port 50502.  Un seul
message est envoyé à chaque 20 millisecondes.  L'adresse est
\texttt{/sponge}.  Elle est suivie de neuf paramètres (tous des entiers):
\texttt{acc1x}, \texttt{acc1y}, \texttt{acc1z}, \texttt{acc2x}, \texttt{acc2y}, \texttt{acc2z}, \texttt{fsr1},
\texttt{fsr2}, \texttt{buttons} .

Les accéléromètres ont des valeurs 16 bits signées, les fsr ont des
valeurs 12 bits non signées, et les boutons sont représentés par les
10 bits les moins significatifs d'un entier à 32 bits.
\end{document}